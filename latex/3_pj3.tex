\chapter{pj3}



專案三: 接續專案二, 各組需對雙輪車進行設計改良,  以提升行進與對戰效能, 各組需採 CAD 進行場景與多輪車零組件設計後, 轉入足球場景中以鍵盤 arrow keys 與 wzas 等按鍵進行控制, 對陣雙方每組將有四名輪車球員, 且每兩人在同一台電腦上操作, 完成後各組需在分組網站中提供所有相關檔案下載連結, 且提供線上分組簡報與分組 pdf 報告連結.\\

球賽計分系統必須採 .ttm 格式建立 (0~99), 使能通用於各類場景計數之用, 並可擴增至三位數計分.\\

除了採用 LED 顯示計分外, 請另外以建立以機械轉盤傳動計分系統 (mechanical counter), 且採 .ttm 格式建立.\\

協同產品設計規格:\\

足球規格 (ball): 白色, 直徑 0.1m, 重量 0.5kg\\

足球場地 (field): 長 4m x 寬 2.5m\\

球門規格 (goal[0] and goal[1]: 長 0.6m, 高 0.3m, 寬 0.1m\\

球員尺寸範圍(player[0]-player[7]: 長寬高各 0.2m, 重量 5kg\\


應交付內容:\\

專案三場景與多輪車零組件設計 (可使用各種 CAD 系統建立, 但必須提供完整的檔案下載連結)\\

專案三控制程式 (以 zmqRemoteAPI Python 製作)\\

專案三開會紀錄與逐字稿 (可利用 jit.si 或 OBS 或其他線上開會系統)\\

專案三各組員任務分配與執行過程影片 (可置於 Youtube 或 Onedrive)\\

專案三網站包括所有協同設計流程所衍生的檔案下載連結, 各檔案必須設法壓縮在 30 MB 內並置於網站downloads 目錄中.\\

專案三線上簡報檔案\\

專案三分組報告 pdf 檔案\\

直接利用 zmqRemoteAPI Python 程式建立場景物件:\\
\begin{lstlisting}[language=Python, frame=single, numbers=left, captionpos=b, basicstyle=\ttfamily\small, showstringspaces=false, breaklines=true, tabsize=4, xleftmargin=15pt]
# zmqRemoteApi_IPv6 為將 zmq 通訊協定修改為 IPv4 與 IPv6 相容
from zmqRemoteApi_IPv6 import RemoteAPIClient
import time
import math
import keyboard
 
# 利用 zmqRemoteAPI 以 23000 對場景伺服器進行連線
client = RemoteAPIClient('localhost', 23000)
# 以 getObject 方法取得場景物件
sim = client.getObject('sim')
box = sim.getObject('/box')
 
# 啟動模擬
sim.startSimulation()
# 建立尺寸數列, 分別定義 x, y, z 方向尺寸
x = 0.2
y = 0.2
z = 0.1
size = [x, y, z]
 
# 利用 size 數列, 建立圓柱物件, 2 代表 cylinder
# 8 表示 respondable, 1 為 質量
digit1_handle = sim.createPureShape(2, 8, size, 1, None)
# 將圓柱物件命名為 digit1, 若用於機械計分可做為個位數轉盤
# 之後可再導入帶有數字組立的外型零件
sim.setObjectAlias(digit1_handle, 'digit1')
# 轉角單位為徑度
sim.setObjectOrientation(digit1_handle, -1, [0, math.pi/2, 0])
# 起始物件中心位於 [0, 0, 0], 為了位於地板, 往 z 提升一個半徑高度
sim.setObjectPosition(digit1_handle, -1, [0, 0, x/2])
 
# 建立 revolute joint 命名為 joint, 且將 joint mode 設為 dynamic, control mode 設為 velocity
joint1_handle = sim.createJoint(sim.joint_revolute_subtype, sim.jointmode_dynamic, 0, None)
sim.setObjectInt32Param(joint1_handle, sim.jointintparam_dynctrlmode, sim.jointdynctrl_velocity)
sim.setObjectAlias(joint1_handle, 'joint1')
 
# 取得 cylinder 的位置座標
digit1_pos = sim.getObjectPosition(digit1_handle, -1)
joint1_pos = [digit1_pos[0], digit1_pos[1], digit1_pos[2]]
 
# 將 joint1 至於 cylinder 中心
sim.setObjectPosition(joint1_handle, -1, joint1_pos)
# 取得 digit1_handle 的方位
digit1_ori = sim.getObjectOrientation(digit1_handle, -1)
# 將 joint1_handle 方位與 digit1 對齊
sim.setObjectOrientation(joint1_handle, -1, digit1_ori)
 
# 將 joint1 置於 box 上
sim.setObjectParent(joint1_handle, box, True)
# 將 cylinder 置於 joint1 上
sim.setObjectParent(digit1_handle, joint1_handle, True)
 
# 鎖定 joint1
sim.setJointForce(joint1_handle, float('inf'))
 
print("基本場景建立完成!")
 
# 設定主迴圈
while True:
    # 設定 joint1 目標速度
    sim.setJointTargetVelocity(joint1_handle, 10)
    # 讓 coppeliasim 有時間按照設定讓 joint1 旋轉
    time.sleep(0.01) 
 
    if keyboard.is_pressed('q'):
        # 可以按下 q 鍵跳出重複執行迴圈
        break
 
# 終止模擬
#sim.stopSimulation()
\end{lstlisting}
